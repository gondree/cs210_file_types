\documentclass[11pt]{article}

\usepackage[pdftex]{graphicx,color}
\usepackage{subfigure}
\usepackage{epstopdf}

\usepackage{amsmath}
\usepackage{amsfonts}
\usepackage{amssymb}
\usepackage{amsthm}

\usepackage{algorithmic}
\usepackage{algorithm}

\usepackage{multicol}
\usepackage{setspace} %for double, 1.5, single spacing etc.
\usepackage{xspace}
\usepackage{aeguill}
\usepackage{verbatim}
\usepackage{url}
\usepackage{times}

\DeclareGraphicsRule{.pdftex}{pdf}{.pdftex}{} % required for combined latex/pdf xfig figures
\DeclareGraphicsRule{.tif}{png}{.png}{`convert #1 `dirname #1`/`basename #1 .tif`.png}

%%%%%%%% BEGIN Editing MACROS %%%%%%%%

% Editing Macros
\newcounter{todo}\setcounter{todo}{0}
\newcommand{\todo}[1]{{\em {\bf TODO \addtocounter{todo}{1}\thetodo:} #1}} 

%changeable conventions in text
\newcommand{\etal}{\emph{et al.}\xspace}
\newcommand{\ie}{\textit{i.e.}\xspace}
\newcommand{\eg}{\textit{e.g.}\xspace}
\newcommand{\deletia}{\ldots [deletia] \ldots}
\newcommand{\snip}{\ldots \<snip\>\ldots}

%%%%%%%% End Editing MACROS %%%%%%%%
%%%%%%%% BEGIN Technical MACROS %%%%%%%%

%custom theorem environments, based on amsthm

\theoremstyle{definition} %plain, definition, remark  (from amsthm package)
\newtheorem{theorem}{Theorem}
\newtheorem{corollary}[theorem]{Corollary}
\newtheorem{lemma}[theorem]{Lemma}
\newtheorem{definition}{Definition}
\newtheorem*{notation}{Notation}
\newtheorem{claim}{Claim}
\newtheorem*{remark}{Remark}

%general
\newcommand{\st}{ \ | \ }                       % shortcut for set-builder notation
\newcommand{\event}[1]{\mathbf{#1}}             % for probabilities, an event
\newcommand{\xor}{\oplus}

%complexity theory
\newcommand{\Class}[1]{\mathbf{\mathsf{#1}}}    	% use like "\Class(NSPACE}(n)" or "\Class(TM) \ M"
\newcommand{\Problem}[1]{\textrm{\textsc{#1}}}

%families, distributions, selection from distribution, etc
\newcommand{\Funfam}[1]{\mathsf{#1}}           % for a family of functions
\newcommand{\Perm}[1]{\Funfam{Perm}({#1})}
\newcommand{\Rand}[2]{\Funfam{Rand}(#1)}
\newcommand{\Randd}[2]{\Funfam{Rand}(#1,#2)}
\newcommand{\getsr}{{\:\stackrel{{\scriptscriptstyle\hspace{0.2em}\$}}{\leftarrow}\:}}

%encryption scheme, etc
\newcommand{\calK}{\mathcal{K}}     % key extraction algorithm
\newcommand{\calE}{\mathcal{E}}     % encryption algorithm
\newcommand{\calD}{\mathcal{D}}     % decryption algorithm
\newcommand{\calV}{\mathcal{V}}	%message authentication verifier
\newcommand{\calT}{\mathcal{T}} 	%message authentication tag generator

%experiment notation
\newcommand{\Prob}[1]{\Pr \left[ #1 \right]}
\newcommand{\ProvOver}[2]{\Pr_{#1} \left[ #2 \right]}
\newcommand{\Adv}{{\mathbf{Adv}}}
\newcommand{\Exp}{{\mathbf{Exp}}}
\newcommand{\prp}{{\mathrm{prp}}}
\newcommand{\outputs}{{\Rightarrow}}
\newcommand{\andthen}{{\::\;\;}}

%game notation
\definecolor{GameHighlight1}{gray}{.85}
\newcommand{\afterbad}[1]{\colorbox{GameHighlight1}{$#1$}}

%common pseudocode
\newcommand{\codestyle}[1]{\textbf{#1}\;\;}
\newcommand{\cif}{\codestyle{if}}
\newcommand{\cthen}{\codestyle{then}}
\newcommand{\celse}{\codestyle{else}}
\newcommand{\celseif}{\codestyle{else if}}
\newcommand{\cdo}{\codestyle{do}}
\newcommand{\cfor}{\codestyle{for}}
\newcommand{\creturn}{\codestyle{return}}


%%%%%%%% END MACROS %%%%%%%%


\setlength{\oddsidemargin}{0in} \setlength{\evensidemargin}{0in} \setlength{\textheight}{9in}
\setlength{\textwidth}{6.5in} \setlength{\topmargin}{-0.5in} \setlength{\headheight}{0.0in}
\setlength{\headsep}{0.0in} \setlength{\parskip}{0.2in} \setlength{\parindent}{0.0in}

\title{\textbf{Assignment 1}\\[2ex] \normalsize{ECS 220 --- Prof. Rogaway --- Winter 2006}}
\date{January 17, 2006}
\author{\begin{tabular}{c}
    Mark Gondree \\
    {\small \texttt{gondree@cs.ucdavis.edu}}
\end{tabular}}

\begin{document}
\maketitle

\section*{Problem 1} Since these machines compute a function $f$, and do not accept or reject, we do not need a set of accepting states $F$. We define the computation to be done, in both cases, when the entire input has been read. Each machine has the form $M = (Q, \Sigma_1, \Sigma_2, \delta, v, q_0)$, where
$$\begin{array}{l}
Q \textrm{ is a finite set of states} \\
q_0 \in Q \textrm{ is the machine's start state} \\
\Sigma_1 \textrm{ is the input alphabet} \\
\Sigma_2 \textrm{ is the output alphabet} \\
\delta: Q \times \Sigma_1 \to Q \textrm{ is the transition function, defined as in a normal DFA} \\
v \textrm{ is defined below, for each case.}
\end{array}$$

\begin{itemize}

\item[A)] 

Let $M = (Q, \Sigma_1, \Sigma_2, \delta, v, q_0)$ be a \emph{Moore Machine}. $Q, \Sigma_1, \Sigma_2, \delta, q_0$ are defined as above, and
$$v: Q \to \Sigma_2 \textrm{ is a function to annotate states with characters from } \Sigma_2$$

$v(q)$ gives the symbol annotating state $q$.

The function $f: \Sigma_1^* \to \Sigma_2^*$ computed by $M$ is the following:
$$ f(x_1 \ldots x_n) = y \textrm{ if } y = v(q_1) \ldots v(q_n) \textrm{ where, for $0<i\leq n$, } \delta(q_{i-1}, x_i) = q_i$$

\item[B)] 

Let $M = (Q, \Sigma_1, \Sigma_2, \delta, v, q_0)$ be a \emph{Mealey Machine}.  $Q, \Sigma_1, \Sigma_2, \delta, q_0$ are defined as above, and
$$v: Q \times \Sigma_1 \to \Sigma_2 \textrm{ is a function to annotate transitions with characters from } \Sigma_2$$

$v(q,x)$ gives the symbol annotating the transition out of $q$ upon reading character $x$.

The function $f: \Sigma_1^* \to \Sigma_2^*$ computed by $M$ is the following:
$$ f(x_1 \ldots x_n) = y \textrm{ if } y=v(q_0, x_1) \ldots v(q_{n-1}, x_n) \textrm{ where, for $0<i \leq n$, } \delta(q_{i-1}, x_i) = q_i$$


\item[C)] 

$f$ is computable by a Mealey machine iff $f$ is computable by a Moore machine.

\begin{proof}\hfill
\begin{itemize}

\item[$\Longleftarrow$] 
Given a Moore machine $A = (Q, \Sigma_1, \Sigma_2, \delta, v, q_0)$ computing $f$, we can build a Mealey machine $B = (Q, \Sigma_1, \Sigma_2, \delta, v' q_0)$ that computes $f$ by defining:
$$v'(q_i, a) = v(q_j) \textrm{ when } \delta(q_i, a) = q_j$$

\item[$\Longrightarrow$] 
Given a Mealey machine $A = (Q, \Sigma_1, \Sigma_2, \delta, v, q_0)$ computing $f$, we can build a Moore machine $B = (Q', \Sigma_1, \Sigma_2, \delta', v', q_0')$ that computes $f$ by defining:
$$\begin{array}{l}
Q' \subseteq \{ \langle q_j,b \rangle \st q_j \in Q, b \in \Sigma_2 \} \\
q_0' = \langle q_0, \varepsilon \rangle \\
v'(\langle q_j,b \rangle) = b \\
\delta'(\langle q_j, b \rangle, a) = \langle \delta(q_j,a), v(q_j,a) \rangle
\end{array}$$

\end{itemize}
\end{proof}

\end{itemize}


\section*{Problem 2}
No -- there are minimal machines with equal numbers of states that accept the same language that cannot be transformed into each other via renaming. Here is an example:

\begin{figure}[h]
\begin{center}\scalebox{.7}{\input{figures/ecs220_hw1_fig_a.pdftex_t}}\end{center}
\caption{Two NFAs with equivalent languages. This is easiest to see by building their regular expressions: $L(M_1) = \{ a^*a, a^*b, a^*bb \}$ and $L(M_2) = \{ aa^*, aa^*b, aa^*bb, b, bb \}$; its clear these are the same.}
\end{figure}

Each compute the same language. Each is minimal: four states is minimal since $b \not \approx_L bb$ (append $b$), $a \not \approx_L b, bb$ (append $a$), $\varepsilon \not \approx_L a, b, bb$ (append $\varepsilon$). Its clear there is no renaming function which can transform $M_1$ into $M_2$ or vice-versa: $M_2$ has an $\varepsilon$-move to the final state, where $M_1$ has no such move.

\pagebreak
\section*{Problem 3}
Let $M = (Q, \Sigma, \delta, q_0, F)$ be a 2-way finite automaton (2WFA), where $\delta: Q \times (\Sigma \cup \{ \rhd, \lhd \}) \to Q \times \{L,R\}$, and $Q, \Sigma, q_0, F$ are defined as in a normal DFA.

I will follow the convention established in class, namely that $x \in L(M)$ when $M$ accepts input $\rhd x \lhd$, and $M$ accepts an input when its head reads the right-marker $\lhd$ when the machine is in a final state $q \in F$. If the machine never does this, it rejects the input.

\begin{definition}[right-left transcript]
Let $T_M^i(x) \in (Q \times \{L,R\})^*$ be the right-left transcript of $M$ under the $i$th character of $x$ on input $\rhd x \lhd$. $T_M^i(x)$ is a list of moves. The $j$th move is $(q,L)$ if $M$'s head was pointing at the $i+1$th input character, is now pointing at the $i$th input character, is in state $q$, and has pointed at this character $j-1$ times before. Alternately, it is $(q,R)$ if its head was pointing at the $i-1$th input character, etc. In short, $T_M^i(x)$ lists the values of $\delta$ that caused $M$ to point to $x$'s $i$th position, in the order of $M$'s execution on $x$. 
\end{definition}

\begin{definition}[right-only transcript]
Let $Right(T_M^i(x)) \in (Q \times \{R\})^*$ be those moves of $T_M^i(x)$ of the form $(q,R)$ for any $q\in Q$. It is an edited right-left transcript, dropping all moves that came from the right. This is the the right-only transcript of $M$ under $i$ on input $x$.
\end{definition}

\begin{figure}[h]
\begin{center}\scalebox{.9}{\input{figures/ecs220_hw1_fig_b.pdftex_t}}\end{center}
\caption{The trace of a 2WFA $M$ on input $x = x_1 \ldots x_n z_1 \ldots z_m$. The right-left transcript $T_M^{n+1}(x)$ is given by reading down the relevant data in the column below character $z_1$. The right-only transcript $Right(T_M^{n+1}(x))$ is given by reading down the data about the circled states in this column.}
\end{figure}

\pagebreak
\begin{lemma}
$\mathcal{L}$ is regular iff $\mathcal{L} = L(M)$ for some 2WFA $M$.
\end{lemma}

\begin{proof}\hfill
\begin{itemize}

\item[$\Longrightarrow$] If $\mathcal{L}$ is regular, then it is accepted by some DFA, and every DFA is (essentially) a 2WFA.

\item[$\Longleftarrow$] If $M$ is a 2WFA, then $\mathcal{L} = L(M)$ is regular: Lemma~\ref{work} claims that every 2WFA partitions $\Sigma^*$ into a finite number of equivalence classes that respect the equivalence $\approx_\mathcal{L}$, and (by the Myhill-Nerode theorem) if $\Sigma^* / \approx_\mathcal{L}$ is finite, then $\mathcal{L}$ is regular.
\end{itemize}\end{proof}

\begin{lemma}\label{work}
Any 2WFA partitions $\Sigma^*$ into a finite number of equivalence classes, that each respect the equivalence $\approx_\mathcal{L}$.
\end{lemma} 

\begin{proof}
Let $M = (Q, \Sigma, \delta, q_0, F)$ be a 2WFA.

Consider $x,y \in \Sigma^*$.  Let $[x] = [y]$ (their classes are the same) if, for all $z \in \Sigma^*$, either $M$ enters an infinite loop on $xz$ and $yz$, or $Right(T_M^{|x|+1}(xz)) = Right(T_M^{|y|+1}(yz))$.

For any $z$, if the right-only transcript is $\varepsilon$ then $M$ has entered an infinite loop before reading the first character of $z$. For any $z$, if the right-only transcript is infintely long, then $M$ has entered an infinite loop. Infinitely long transcripts can be considered 0-length transcripts, a special finite-length transcript. So, we only need to consider finite-length right-only transcripts.

So, $[x] \neq [y]$ in the case that some $z$ causes two finite, unequal, right-only transcripts. If $q \in Q$ appears twice in any right-transcript, then $M$ will enter an infinite loop and the transcript will not be finite. The number of finite right-only transcripts is bound by $(|Q|+1)! \geq \sum_{i=0}^{|Q|}{ |Q|! / i! }$, the sum of the ways to order $|Q|-i$ objects drawn from a set of size $|Q|$. This is finite, thus there are a finite number of different classes.

If $x_1 \in [x]$ and $x_2 \in [x]$, then $x_1 \approx_\mathcal{L} x_2$. By definition, any $z$ causes $x_1z$ and $x_2z$ to produce the same right-transcripts. By definition, $x_1 \approx_\mathcal{L} x_2$ means, for any $z$, $x_1z \in \mathcal{L} \iff x_2z \in \mathcal{L}$. In fact, if the last moves in the two right-only transcripts agree, then $x_1z$ and $x_2z$ will either both be accepted or both be rejected (since, at this point, $M$ never again moves far enough left to read the characters of $x_1$ or $x_2$, so its accept/reject behavior is entirely based on the last state of the right-only transcript and $z$, which are identical for $x_1z$ and $x_2z$). 

Thus, $\#\{ [x] \st x \in \Sigma^* \}$ is finite and members of $[x]$ respect the $\approx_\mathcal{L}$ equivalence.
\end{proof}

\pagebreak
\section*{Problem 4}
If $L \subseteq 1^*$ then $L^*$ is regular.

\begin{proof}\hfill
\begin{itemize}

\item[(1)] If $L$ is finite, then it is regular and $L^*$ is regular.
\item[(2)] By lemma~\ref{coprime}, if $L$ is infinite and contains two words of coprime lengths, then $L^*$ is regular.
\item[(3)] By lemma~\ref{noncoprime}, if $L$ is infinite and every word of $L$ has a length that is divisible by some integer $m>1$, then $L^*$ is regular.
\item[(4)] By lemma~\ref{nonexist}, no other cases exist.

\end{itemize}\end{proof}

\begin{lemma}\label{coprime}
If $L\subseteq 1^*$ is infinite and contains two words of coprime lengths, then there is some sufficiently large $c_0$ such that all words of length at least $c_0$ are in $L^*$. So, any word in $1^* - L^*$ has a length less than $c_0$. There are a finite number of such unary words. So, $1^* - L^*$ is finite, and therefore regular. The complement of any regular language is regular, so $L^*$ is regular
\end{lemma}

\begin{proof}
Take $1^p, 1^q \in L$ such that $\mathrm{gcd}(p,q)=1$. Thus, $\{ p^i \mod{q} \st 0 \leq i < q\}$ contains all residues. Any $c$ has some residue, $c \equiv x \mod{q}$. For some $i$, $p^i \equiv x \mod{q}$. Let $c=x+nq$ and $p^i = x+mq$. For $c_0=q+p^{q-1}$, any $c > c_0$ can be represented as $c = p^i + (n-m)q$.

Because $c = p^i + (n-m)q$, we can form $1^c$ by concatenating $(n-m)$ copies of $1^q$ to $p^{i-1}$ copies of $p$. Thus, for all $c > c_0, 1^c \in L^*$.
\end{proof}

\begin{lemma}\label{noncoprime}
If $L$ is infinite and every word in $L$ has a length that is divisible by some integer $m>1$, then $L^*$ is regular.
\end{lemma}

\begin{proof}
Let $S = \{ x \in \mathbb{N} \st 1^x \in L \}$, the set of lengths of words in $L$. Take $a,b$ to be the smallest two elements in $S$. If $\mathrm{gcd}(a,b)=n$ then $n | x$ for any $x \in S$, by the premise. Also, it is clear that if $1^y \in L^*$ then $n | y$.

Let $S_n = \{ x/n \st x \in S \}$. $S_n$ has two coprime elements, $a/n$ and $b/n$. Let $L_n = \{ 1^y \st y \in S_n \}$. By lemma~\ref{coprime}, $L_n^*$ is regular.

Thus, $1^x \in L^*$ if $x \equiv 0 \mod{n}$ and $1^{x/n} \in L_n^*$. This is a regular language --- simply take the DFA accepting $L_n^*$ and add a chain of $n-1$ non-accepting states between every two states in the original machine.
\end{proof}

\begin{lemma}\label{nonexist}
If $L$ is not finite, then either there are two words $1^p, 1^q \in L$ such that $\mathrm{gcd}(p,q)=1$ or there is some integer $m>1$ such that $m | x$ for all $1^x \in L$.
\end{lemma}

\begin{proof}
Let $S = \{ x \in \mathbb{N} \st 1^x \in L \}$, the set of lengths of words in $L$. Assume that for any two $x,y \in S$, $\mathrm{gcd}(x,y) > 1$. Also assume that there is no integer $m>1$ such that $m | x$ for all $x \in S$.

Since there is no $m$ that divides every element in $S$, there must be some $z \in S$ such that $\mathrm{gcd}(x,y)$ does not share any factor $m$ with $z$ (the only other option being that every $z$ shares a factor with $\mathrm{gcd}(x,y)$ but none of these factors are the same, which means $\mathrm{gcd}(x,y)$ has an infinite number of factors which is impossible for any finite number). Thus, $\mathrm{gcd}(\mathrm{gcd}(x,y),z) = 1$. So either $\mathrm{gcd}(x,z)=1$ or $\mathrm{gcd}(y,z)$ = 1, which is a contradiction.
\end{proof}

\end{document}
